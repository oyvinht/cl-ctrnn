\documentclass[english]{article}
\usepackage[]{acronym}
\usepackage[]{charter}
\usepackage[]{fullpage}
\usepackage[]{listings}
\usepackage[utf-8]{inputenc}

\setlength{\parindent}{0pt}
\setlength{\parskip}{12pt}

\begin{document}

\part{Conventions}
\section{Acronyms}

\begin{acronym}
\acro{ASDF}{Another System Definition Facility}
\acro{CTRNN}{Continuous Time Recurrent Neural Network}
\end{acronym}

\section{Licensing}

\section{Naming}
All accessors are prefixed by class name and ``-'', e.g. ``neuron-membrane-potential''. All destructive functions and methods are suffixed by an exclamation mark, e.g. ``add-neuron!''.

\newpage
\part{Usage}
\newpage
\section{Loading}
\ac{ASDF} is used for loading the library. Assuming that \ac{ASDF} is loaded, ac{CL-CTRNN} can be loaded by:

\begin{lstlisting}
(push \#P"/path/to/cl-ctrnn'' asdf:*central-registry*)
(asdf:oos 'asdf:load-op 'cl-ctrnn)
\end{lstlisting}

\section{Examples}


\newpage
\part{Classes}
\newpage
\section{neural-network}
\label{neural-network}

\subsection{Description}
A \ac{CTRNN} that simulates a real neural network. Contains neurons(section \ref{neuron}) that may be interconnected with synapses(section \ref{synapse}) and/or connected to external sensors and/or motors.

\subsection{Superclasses}
None.

\subsection{Known subclasses}
None.

\subsection{Specific methods}

\subsubsection{Accessors}
neural-network-neurons\\

\subsubsection{Other}
add-neuron! (neural-network neuron)\\
synchronously-update-membrane-potentials! (neural-network)\\


\newpage
\section{neuron}
\label{neuron}

\subsection{Description}
Representation of a single neuron.

\subsection{Superclasses}
None.

\subsection{Known subclasses}
motor-neuron(section \ref{motor-neuron}), sensor-neuron(section \ref{sensor-neuron}).

\subsection{Specific methods}

\subsubsection{Accessors}
neuron-bias\\
neuron-dendrites\\
neuron-external-current\\
neuron-external-current-magnitude\\
neuron-maximum-membrane-potential\\
neuron-membrane-potential\\
neuron-snapshot-firing-frequency\\
neuron-time-constant\\

\subsubsection{Other}
add-dendrite! (neuron synapse)\\
firing-frequency (neuron)\\
update-membrane-potential! (neuron)\\


\newpage
\section{motor-neuron}
\label{motor-neuron}

\subsection{Description}
A neuron that will call an external motor-function after updating the membrane-potential.

\subsection{Superclasses}
neuron(\ref{neuron})

\subsection{Known subclasses}
None.

\subsection{Specific methods}
\subsubsection{Accessors}
motor-neuron-motor-function\\

\subsubsection{Other}
None.\\


\newpage
\section{sensor-neuron}
\label{sensor-neuron}

\subsection{Description}
A neuron that will call an external sensor function before updating membrane-potential.

\subsection{Superclasses}
neuron(\ref{neuron})

\subsection{Known subclasses}
None.

\subsection{Specific methods}
\textbf{Accessors}
sensor-neuron-sensor-function\\

\textbf{Other}
None.\\

\section{synapse}
\label{synapse}

\subsection{Description}
A weighted connection between two neurons. Neurons keep synapses in dendrites(section \ref{neuron}), and each synapse knows only its from-neuron.

\subsection{Superclasses}
None.

\subsection{Known subclasses}
None.

\subsection{Methods}

\subsubsection{Accessors}
synapse-from-neuron\\
synapse-strength\\

\subsubsection{Other}
add-dendrite! (neuron, synapse)\\

\end{document}